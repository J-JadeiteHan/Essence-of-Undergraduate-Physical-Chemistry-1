\usepackage{kotex}
\usepackage{amsthm, thmtools}
\usepackage{ hyperref,%\autoref % n.b. \Autoref is defined by thmtools
cleveref,% \cref % n.b. cleveref after! hyperref
}

\usepackage[framemethod=TikZ]{mdframed}
\mdfsetup{skipabove=2em,skipbelow=2em}

\theoremstyle{definition}
\declaretheoremstyle[
    headfont=\sffamily\bfseries, bodyfont=\normalfont,
    % mdframed={
    %     linewidth=2pt,
    %     rightline=true, topline=false, bottomline=false,
    %     linecolor=black, backgroundcolor=black!3!white,
    % }
]{thmbox}

\declaretheoremstyle[
    headfont=\sffamily\color{black!70!black}, bodyfont=\normalfont,
    numbered=no,
    % mdframed={
    %     linewidth=0pt,
    %     rightline=false, topline=false, bottomline=false,
    %     linecolor=black, backgroundcolor=black!2!white,
    % },
    qed=\qedsymbol
]{thmproofbox}

\theoremstyle{definition}
\declaretheoremstyle[
    headfont=\bfseries\sffamily\color{black!70!black}, bodyfont=\normalfont,
    % mdframed={
    %     linewidth=0pt,
    %     rightline=false, topline=false, bottomline=false,
    %     linecolor=black, backgroundcolor=black!3!white,
    % }
]{exerbox} 

\declaretheorem[numberwithin=section,style=thmbox, name=정의, refname={정의, 정의}, Refname={정의, 정의}]{defn}
\declaretheorem[sibling=defn,style=thmbox, name=예시, refname={예시,예시}, Refname={예시,예시}]{eg}
\declaretheorem[sibling=defn,style=thmbox, name=명제, refname={명제, 명제}, Refname={명제, 명제}]{prop}
\declaretheorem[sibling=defn,style=thmbox, name=정리, refname={정리, 정리}, Refname={정리, 정리}]{thm}
\declaretheorem[sibling=defn,style=thmbox, name=도움정리, refname={도움정리, 도움정리}, Refname={도움정리, 도움정리}]{lem}
\declaretheorem[sibling=defn,style=thmbox, name=따름정리, refname={따름정리, 따름정리}, Refname={따름정리, 따름정리}]{cor}
\declaretheorem[sibling=defn,style=thmbox, name=법칙, refname={법칙, 법칙}, Refname={법칙, 법칙}]{law}
\declaretheorem[sibling=defn,style=thmbox, name=관찰, refname={관찰, 관찰}, Refname={관찰, 관찰}]{obs}
\declaretheorem[sibling=defn,style=thmbox, name=주의, refname={주의, 주의}, Refname={주의, 주의}]{warn}
\declaretheorem[sibling=defn,style=thmbox, name=질문, refname={질문, 질문}, Refname={질문, 질문}]{que}
\declaretheorem[sibling=defn,style=thmbox, name=사실, refname={사실, 사실}, Refname={사실, 사실}]{fact}
\declaretheorem[sibling=defn,style=thmbox, name=논의, refname={논의, 논의}, Refname={논의, 논의}]{rem}
\declaretheorem[sibling=defn,style=thmbox, name=표기, refname={표기, 표기}, Refname={표기, 표기}]{notn}
\declaretheorem[sibling=defn,style=exerbox, name=연습문제, refname={연습문제, 연습문제}, Refname={연습문제, 연습문제}]{exr}
\declaretheorem[sibling=defn,style=exerbox, name=*연습문제, refname={*연습문제, *연습문제}, Refname={*연습문제, *연습문제}]{exrhard}

\declaretheorem[style=thmbox, name=주장, refname={주장, 주장}, Refname={주장, 주장}]{claim}
\declaretheorem[style=thmbox, numbered=no, name=주장, refname={주장, 주장}, Refname={주장, 주장}]{claimnonumbered}

\declaretheorem[style=thmbox, numbered=no, name=논의, refname={논의, 논의}, Refname={논의, 논의}]{remnonum}

\declaretheorem[style=thmproofbox, name=증명]{replacementproof}
\renewenvironment{proof}[1][]{\begin{replacementproof}[name=#1]}{\end{replacementproof}}

\usepackage[most]{tcolorbox}

\newenvironment{sketch}[1][\textbf{\textsf{[연습장]}}]
{\begin{tcolorbox}[boxrule=0.1mm, breakable]
\begin{center}
#1
\end{center}
}
{\end{tcolorbox}}
