\usepackage[b5paper,
    inner=16mm,         % Inner margin
    outer=24mm,         % Outer margin
    bindingoffset=10mm, % Binding offset
    top=28mm,           % Top margin
    bottom=28mm,        % Bottom margin
]{geometry}
\usepackage[utf8]{inputenc}
\usepackage{amsmath, amsthm, amsfonts, kotex, enumitem, setspace, lipsum, amssymb, tabularray, mathtools, chngcntr, tikz, tikz-cd, graphicx, scalerel, graphbox, stmaryrd, float, colortbl, multirow, multicol, caption, subcaption, pgffor, appendix, makeidx, physics, gensymb, footnote, cancel}
\usepackage[colorlinks=true,linkcolor=blue,citecolor=blue]{hyperref}
% \let\Bbbk\relax
% \usepackage{newtxtext, newtxmath}
\usepackage{libertineRoman, libertineMono}
\usepackage[T1]{fontenc}
\usepackage[notextcomp]{stix}
\setstretch{1.4}
\usepackage[symbol]{footmisc}
% \usepackage[bottom]{footmisc}
% \usepackage{microtype}
% \usepackage{silence}

% % Ignore all warnings about font size substitutions
% \WarningFilter{font}{Font shape `T1/cmss/m/n' in size}

\usepackage{iftex}
\ifPDFTeX
    \usepackage{dhucs-nanumfont}
    % \usepackage{newtxmath}
\else\ifXeTeX
      \usepackage{fontspec}
      \setmainhangulfont{NanumMyeongjo}[%
    	Renderer = OpenType,
    	FontFace = {m}{n}{ Font = * },
    	FontFace = {m}{it}{ Font = *, FakeSlant=.167 },
    	FontFace = {m}{up}{ Font = * },
    	FontFace = {bx}{n}{ Font = {* ExtraBold} },
    	FontFace = {bx}{it}{ Font = {* ExtraBold}, FakeSlant=.167 }
    ]
    \setsanshangulfont{NanumGothic}
\else\ifLuaTeX
  \usepackage{fontspec}
      \setmainfont{Times New Roman}
      \setmainhangulfont{NanumMyeongjo}[%
    	Renderer = OpenType,
    	FontFace = {m}{n}{ Font = * },
    	FontFace = {m}{it}{ Font = *, FakeSlant=.167 },
    	FontFace = {m}{up}{ Font = * },
    	FontFace = {bx}{n}{ Font = {* ExtraBold} },
    	FontFace = {bx}{it}{ Font = {* ExtraBold}, FakeSlant=.167 }
    ]
    \setsanshangulfont{NanumGothic}
\fi\fi\fi


% Upper cases

\foreach \x in {A,...,Z}{%
\expandafter\xdef\csname bb\x\endcsname{\noexpand\ensuremath{\noexpand\mathbb{\x}}}
\expandafter\xdef\csname cal\x\endcsname{\noexpand\ensuremath{\noexpand\mathcal{\x}}}
\expandafter\xdef\csname frak\x\endcsname{\noexpand\ensuremath{\noexpand\mathfrak{\x}}}
\expandafter\xdef\csname bf\x\endcsname{\noexpand\ensuremath{\noexpand\mathbf{\x}}}
}

\newcommand{\NN}{\bbN}
\newcommand{\ZZ}{\bbZ}
\newcommand{\QQ}{\bbQ}
\newcommand{\RR}{\bbR}
\newcommand{\CC}{\bbC}
\newcommand{\FF}{\bbF}

% Lower cases

\foreach \x in {a,...,z}{%
\expandafter\xdef\csname frak\x\endcsname{\noexpand\ensuremath{\noexpand\mathfrak{\x}}}
\expandafter\xdef\csname bf\x\endcsname{\noexpand\ensuremath{\noexpand\mathbf{\x}}}
}


% Brackets

\newcommand{\inner}[2]{\left\langle#1, #2\right\rangle}

\newcommand{\gen}[1]{\left\langle#1\right\rangle}

\newcommand{\floor}[1]{\left\lfloor#1\right\rfloor}
\newcommand{\ceil}[1]{\left\lceil#1\right\rceil}

% Operators

\DeclareMathOperator{\difform}{d\!}

% Shortcuts

\newcommand{\vn}{\varnothing}

\newcommand{\eps}{\varepsilon}
\newcommand{\dt}{\delta}
\newcommand{\one}{\mathbf{1}}

\newcommand{\xR}{\RR^{\star}}

\newcommand{\seq}[2][\NN]{(#2_n)_{n \in #1}}
\newcommand{\subseq}[1]{(#1_{n(k)})_{k \in \NN}}
\newcommand{\sub}{\subseteq}
\newcommand{\pa}{\partial}
\newcommand{\setm}{\setminus}

\newcommand{\lop}{L'H\^opital}
\newcommand{\sch}{Schr\"odinger}

\newcommand{\toto}{\rightrightarrows}
\newcommand{\supnorm}[1]{\norm{#1}_{\sup}}

\newcommand{\longto}{\longrightarrow}


\newenvironment{enum}
	{\begin{enumerate}[leftmargin=*, noitemsep, topsep=2pt, label={\sf (\alph*)}]}
	{\end{enumerate}}

\newenvironment{enumroman}
	{\begin{enumerate}[leftmargin=*, noitemsep, topsep=2pt, label={\sf (\roman*)}]}
	{\end{enumerate}}

\numberwithin{equation}{section}

\usepackage{kotex}
\usepackage{amsthm, thmtools}
\usepackage{ hyperref,%\autoref % n.b. \Autoref is defined by thmtools
cleveref,% \cref % n.b. cleveref after! hyperref
}

\usepackage[framemethod=TikZ]{mdframed}
\mdfsetup{skipabove=2em,skipbelow=2em}

\theoremstyle{definition}
\declaretheoremstyle[
    headfont=\sffamily\bfseries, bodyfont=\normalfont,
    % mdframed={
    %     linewidth=2pt,
    %     rightline=true, topline=false, bottomline=false,
    %     linecolor=black, backgroundcolor=black!3!white,
    % }
]{thmbox}

\declaretheoremstyle[
    headfont=\sffamily\color{black!70!black}, bodyfont=\normalfont,
    numbered=no,
    % mdframed={
    %     linewidth=0pt,
    %     rightline=false, topline=false, bottomline=false,
    %     linecolor=black, backgroundcolor=black!2!white,
    % },
    qed=\qedsymbol
]{thmproofbox}

\theoremstyle{definition}
\declaretheoremstyle[
    headfont=\bfseries\sffamily\color{black!70!black}, bodyfont=\normalfont,
    % mdframed={
    %     linewidth=0pt,
    %     rightline=false, topline=false, bottomline=false,
    %     linecolor=black, backgroundcolor=black!3!white,
    % }
]{exerbox} 

\declaretheorem[numberwithin=section,style=thmbox, name=정의, refname={정의, 정의}, Refname={정의, 정의}]{defn}
\declaretheorem[sibling=defn,style=thmbox, name=예시, refname={예시,예시}, Refname={예시,예시}]{eg}
\declaretheorem[sibling=defn,style=thmbox, name=명제, refname={명제, 명제}, Refname={명제, 명제}]{prop}
\declaretheorem[sibling=defn,style=thmbox, name=정리, refname={정리, 정리}, Refname={정리, 정리}]{thm}
\declaretheorem[sibling=defn,style=thmbox, name=도움정리, refname={도움정리, 도움정리}, Refname={도움정리, 도움정리}]{lem}
\declaretheorem[sibling=defn,style=thmbox, name=따름정리, refname={따름정리, 따름정리}, Refname={따름정리, 따름정리}]{cor}
\declaretheorem[sibling=defn,style=thmbox, name=법칙, refname={법칙, 법칙}, Refname={법칙, 법칙}]{law}
\declaretheorem[sibling=defn,style=thmbox, name=관찰, refname={관찰, 관찰}, Refname={관찰, 관찰}]{obs}
\declaretheorem[sibling=defn,style=thmbox, name=주의, refname={주의, 주의}, Refname={주의, 주의}]{warn}
\declaretheorem[sibling=defn,style=thmbox, name=질문, refname={질문, 질문}, Refname={질문, 질문}]{que}
\declaretheorem[sibling=defn,style=thmbox, name=사실, refname={사실, 사실}, Refname={사실, 사실}]{fact}
\declaretheorem[sibling=defn,style=thmbox, name=논의, refname={논의, 논의}, Refname={논의, 논의}]{rem}
\declaretheorem[sibling=defn,style=thmbox, name=표기, refname={표기, 표기}, Refname={표기, 표기}]{notn}
\declaretheorem[sibling=defn,style=exerbox, name=연습문제, refname={연습문제, 연습문제}, Refname={연습문제, 연습문제}]{exr}
\declaretheorem[sibling=defn,style=exerbox, name=*연습문제, refname={*연습문제, *연습문제}, Refname={*연습문제, *연습문제}]{exrhard}

\declaretheorem[style=thmbox, name=주장, refname={주장, 주장}, Refname={주장, 주장}]{claim}
\declaretheorem[style=thmbox, numbered=no, name=주장, refname={주장, 주장}, Refname={주장, 주장}]{claimnonumbered}

\declaretheorem[style=thmbox, numbered=no, name=논의, refname={논의, 논의}, Refname={논의, 논의}]{remnonum}

\declaretheorem[style=thmproofbox, name=증명]{replacementproof}
\renewenvironment{proof}[1][]{\begin{replacementproof}[name=#1]}{\end{replacementproof}}

\usepackage[most]{tcolorbox}

\newenvironment{sketch}[1][\textbf{\textsf{[연습장]}}]
{\begin{tcolorbox}[boxrule=0.1mm, breakable]
\begin{center}
#1
\end{center}
}
{\end{tcolorbox}}


% \usepackage{tablefootnote} 
% \makeatletter 
% \AfterEndEnvironment{mdframed}{%
%  \tfn@tablefootnoteprintout% 
%  \gdef\tfn@fnt{0}% 
% }
% \makeatother 

\newcommand{\tablefootnote}{\footnote}

\allowdisplaybreaks