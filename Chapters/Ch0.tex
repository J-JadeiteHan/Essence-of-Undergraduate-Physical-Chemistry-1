\section{들어가며}
        \hspace{\parindent} 이 문서는 물리화학의 기본 개념을 정리하고, 그 개념들이 어떻게 유도되는지 정리하기 위해 작성되었습니다.
        이 시리즈는 총 3개의 권으로 구성되며, 1권에서는 열역학을 정리합니다. Atkins' Physical Chemistry 11판을 기준으로,
        이 부분은 챕터 1부터 6까지에 해당합니다.
        \par 2권에서는 양자화학을 정리합니다. 이 부분은 챕터 7부터 11까지, 그리고 챕터 16에 해당합니다. 3권에서는 통계물리와 반응속도론,
        고체화학 및 표면화학에 이르는 넓은 범위를 정리합니다. 이 부분은 챕터 12부터 15까지, 그리고 챕터 17부터 19까지에 해당합니다.
        \par 본 책에서는 열역학에 대해 정리합니다. 먼저 1단원에서 기체에 대해 다루고, 2단원에서 열역학 제1 법칙을 다룹니다. 
        3단원에서는 열역학 제2 및 제3 법칙을 다루면서 단일 상황에 대한 열역학 과정을 다루게 됩니다.
        \par 이후 4단원 및 5단원에서는 상평형도에 대해 
        다룹니다. 구체적으로 4단원에서는 단일 물질의 상평형도를, 5단원에서는 혼합물의 상평형도를 다룹니다. 마지막으로 6단원에서는 화학 평형과 
        전기화학 기초를 다룹니다.
\section{에너지에 관하여}
    \hspace{\parindent} 화학에서 에너지의 출입은 \textbf{열역학}의 언어로 기술된다. 이는 거시적인 물체에 적용되며, 3개의 법칙으로 기술된다. 
    단일 원자나 분자의 에너지에 대해서는 \textbf{양자역학}의 언어로 기술된다. 이에 대해서는 2권에서 더 자세히 다룰 예정이다. 
    입자의 운동에 관여하는 에너지는 양자화(Quantized) 되어 있으며, 병진 운동(Translation), 회전 운동(Rotation), 진동 운동(Vibration)의 
    세 가지 운동이 존재한다. 병진 운동 에너지의 준위 간 차이가 매우 작아($\sim 10^{-23}$ eV) 거의 연속적이다. 또한 
    회전 운동의 에너지 준위 간 차이는 약 0.001 eV, 진동 운동에서는 약 0.1 eV, 전자의 에너지 준위 간 차이는 수$\sim$수십 eV이다.
    \par 단일 입자의 에너지와 열역학 사이에는 \textbf{통계역학}이라는 다리가 존재한다. 입자의 분포 $N_i$와 에너지 $\varepsilon_{i}$, 
    절대온도 $T$ 사이에는 다음과 같은 관계가 성립한다:
    \begin{equation*}
        N_i \propto e^{- \frac{\varepsilon_{i}}{kT}}
    \end{equation*}
    이때 $k$는 \textit{볼츠만 상수(Boltzmann's constant)}라 부르며, 그 값은 $1.380 \ 6488 \times 10^{-23}$ \textrm{J K$^{-1}$}에 해당한다. 이러한 
    분포를 \textit{볼츠만 분포(Boltzmann distribution)}라 한다. 이에 대해서는 3권에서 더 자세히 다룰 예정이다.
\pagebreak