\chapter{열역학 제1 법칙}
    \section{내부 에너지}\label{intener}
            \hspace{\parindent} 열역학에서는 우주를 두 부분으로 나눈다:
            \begin{defn}[계와 주위]
            \textbf{계(System)}는 우리가 관심을 갖는 부분이다. 
            \textbf{주위(Surroundings)}는 계를 제외한 나머지 모든 부분이다.
            \end{defn}
            계는 다시 세 종류로 나눌 수 있다:
            \begin{defn}[계의 종류]
            \begin{enum}
            \item \textbf{열린계(Open system)}는 주위와 물질 및 에너지를 교환할 수 있는 계이다. 
            \item \textbf{닫힌계(Closed system)}는 주위와 에너지를 교환할 수 있으나, 물질은 교환할 수 없는 계이다.
            \item \textbf{고립계(Isolated system)}는 주위와 물질 및 에너지를 모두 교환할 수 없는 계이다.
            \end{enum}
            \end{defn}
            \par 열역학에서는 에너지를 일과 열의 형태로 교환한다. 
            \begin{defn}[일]
            열역학에서 \textbf{일(Work)}은 압력에 \underline{대항하여} 움직이는 것을 의미한다.\footnote[1]{%
            물리학에서는 계가 일을 하는 것으로 본다.}
            \end{defn}
            \begin{defn}[에너지]
            열역학에서 계의 \textbf{에너지(Energy)}는 일을 할 수 있는 능력을 의미한다.
            \end{defn}
            계와 주위의 온도가 다르면 \textbf{열(Heat)}의 형태로 에너지가 이동한다. 열을 흐를 수 있게 하는 경계를 \textbf{투열성(Diathermic)}이 있다고 하고, 
            열을 차단하는 경계를 \textbf{단열성(Adiabatic)}이 있다고 한다.\\
            화학 반응에서, \textbf{발열 과정(Exothermic process)}은 열로 에너지를 방출하는 열역학 과정을 의미한다. 예를 들어 연소 반응이 있다. 
            반면 \textbf{흡열 과정(Endothermic process)}은 열 형태의 에너지를 흡수하는 열역학 과정을 의미한다. 예를 들어 기화 반응이 있다. 
            고립계에서 발열 과정이 일어나면 계의 온도가 상승하고, 반대로 흡열 과정이 일어나면 계의 온도가 하강한다.
            \par 분자적 관점에서 보면, 열은 분자의 무작위적인 운동인 \textbf{열운동(Thermal motion)}의 정도를 변화시킨다. 반면 일은 
            분자가 일정한 방향으로 움직이는 것으로 볼 수 있다.
            \par \begin{defn}[내부 에너지]
                \textbf{내부 에너지(Internal energy)}는 계가 가지고 있는 총 에너지를 일컫는다. 기호로 $U$로 표현하고, 계 \underline{내부}의 입자들의 운동 에너지 
                및 퍼텐셜 에너지의 총합이다.\footnote[2]{계 자체의 운동으로 인한 운동 에너지는 포함하지 않는다.} 내부 에너지는 상태 함수이므로, $\Delta U = U_f - U_i$
            로 나타낼 수 있다.
            \end{defn}
            \begin{defn}[상태 함수]
            \textbf{상태 함수(State function)}는 계의 상태에만 의존하고 경로에 의존하지 않는 성질을 일컫는다.
            \end{defn}
            또한 내부 에너지는 
            \textbf{크기 성질(Extensive property)}\footnote[3]{이와 상대되는 개념으로 \textbf{세기 성질(Intensive property)}이 있다. 
            세기 성질은 계의 크기에 무관하다. 압력, 온도 등이 있다.}이다. 즉, 계의 물질량에 의존한다. 따라서 몰 내부 에너지 $U_m = U/n$를 생각할 수 있다. 
            내부 에너지의 단위는 J( = kg m$^2$ s$^{-2}$)이고, 몰 내부 에너지의 단위는 J mol$^{-1}$이다.
            \par 분자는 $3N$의 자유도를 가진다. \textbf{등분배 법칙(Equipartition theorem)}에 따르면, 자유도 하나 당 $\displaystyle\frac{1}{2}kT$의 내부 에너지를 가진다. 
            계의 내부 에너지가 증가하면 더 높은 에너지 준위에 더 많은 수의 분자가 존재하게 된다. 또한, 이상 기체의 내부 에너지는 부피에 무관하다. 이는 분자 간 상호작용이
            존재하지 않기 때문이다.
            \par 열역학 제1 법칙은 다음과 같이 기술된다(단, $q$는 이동한 열 에너지, $w$는 일)\footnote[4]{%
            물리학에서는 $\Delta U = q - w$로 기술한다.}:
            \begin{law}[열역학 제1 법칙]
                \begin{equation*}
                    \Delta U = q + w
                \end{equation*}
            \end{law}
            고립계에서는 $q=0$, $w=0$이므로 내부 에너지 변화량은 $0$이다. 또한, 열역학 제1 법칙은 또한 다음과 같이 기술된다\footnote[5]{%
            이 또한 물리학에서는 $\difform U = \delta q + \delta w$로 기술한다.}:
            \begin{law}[열역학 제1 법칙, 미분형]
                \begin{equation*}
                    \difform U = \delta q + \delta w
                \end{equation*}
            \end{law}
            여기에서 $\difform\, $는 상태 함수에, $\delta$는 경로 함수에 적용한다.
            \par 일은 다음과 같이 정의된다:
            \begin{defn}[일]
                \begin{equation*}
                    \difform w = - \left\vert F \right\vert \difform z
                \end{equation*}
            \end{defn}
            이때 벽면의 단면적을 $A$라 하면, $\left\vert F \right\vert = p_{\mathrm{ex}}A$가 성립하므로 다음이 성립한다\footnote[6]{%
            물리학에서는 계에 관심을 두기 때문에, $\difform w = p \difform V$로 정의한다.
            }:
                \begin{align*}
                        \difform w &= - p_{\mathrm{ex}}A \difform z\\
                        &= -p_{\mathrm{ex}} \difform V
                \end{align*}
            이외에도, 계에 가해지는 일은 다음과 같은 것들이 있을 수 있다:
                \begin{table}[H]
                \centering
                    \begin{tabular}{ c c c c }
                        \hline
                        \rowcolor{lightgray}
                        일의 종류 & $\difform w$ & 설명 & 단위 \\
                        \hline
                        팽창 & $-p_{\mathrm{ex}}\difform V$ & $p_{\mathrm{ex}}$: 외부 압력 & Pa\\
                        & & $\difform V$: 부피 변화 & m$^3$ \\
                        \hline
                        표면 팽창 & $\gamma \difform \sigma$ & $\gamma$: 표면 장력 & N m$^{-1}$\\
                        & & $\difform \sigma$: 표면적 변화 & m$^2$\\
                        \hline
                        연장 & $f\difform l$ & $f$: 장력 & N\\
                        & & $\difform l$: 길이 변화& m\\
                        \hline
                        전기적 일 & $\phi \difform Q$ &$\phi$: 전기 퍼텐셜&V\\
                        & & $\difform Q$: 전하량 변화&C\\
                        \hline
                        전기적 일 & $Q \difform \phi$ &$Q$: 전하량&C\\
                        & &$\difform \phi$: 전기 퍼텐셜 변화&V\\
                        \hline
                    \end{tabular}
                \end{table}
            따라서, 계에 팽창으로써 가한 총 일은 다음과 같이 구할 수 있다:
            \begin{law}[팽창으로 인한 일]
                \begin{equation*}
                    w = -\int_{V_i}^{V_f}p_{\mathrm{ex}}\difform V
                \end{equation*}
            \end{law}
            전기적 일 등과 같은 \textbf{비팽창 일(Non-expansion work)} (또는 \textbf{추가 일(Additional work)}) 또한 세기 성질과 
            크기 성질의 곱으로 유사하게 구할 수 있다. \\
            일정한 압력에 의한 일은 다음과 같이 구할 수 있다:
                \begin{equation*}
                    w = -p_{\mathrm{ex}}\left(V_f - V_i \right) = -p_{\mathrm{ex}}\Delta V
                \end{equation*}
            \textbf{자유 팽창(Free expansion)}은 외부 압력 $p_\mathrm{ex} = 0$일 때이므로, 계에 가한 일은 $w = 0$이다.
            \par \begin{defn}[가역 변화]
            열역학에서 \textbf{가역 변화(Reversible change)}는 열역학적 변수에 \textbf{매우 작은(Infinitesimal)} 변화를 가하여 역으로 되돌릴 수 있는 
            반응을 의미한다.
            \end{defn}
            \begin{defn}[열평형과 역학적 평형]
            \begin{enum}
            \item \textbf{열평형(Thermal equilibrium)}은 온도가 같은 두 계가 이루는 가역적인 평형이다.
            \item \textbf{역학적 평형(Mechanical equilibrium)}은 압력이 같은 두 계가 이루는 가역적인 평형이다.
            \end{enum}
            \end{defn}
            외부 압력이 내부 압력과 같을 때(즉, 계와 주위가 역학적 평형을 이룰 때), 다음이 성립한다:
            \begin{align*}
                \difform w &= -p \difform V\\
                w &= - \int_{V_i}^{V_f} p \difform V
            \end{align*}
            계가 등온 가역 팽창 과정을 거친다고 하자. 이때 이상 기체 상태방정식 $pV = nRT$에서, $p = nRT / V$로 치환하면 다음과 같은 결과를 얻는다:
            \begin{obs}[등온 가역 과정]
                \begin{equation*}
                    w = -nRT\int_{V_i}^{V_f} \frac{\difform V}{V} = -nRT \ln{\frac{V_f}{V_i}}
                \end{equation*}
            \end{obs}
            $V_i > V_f$일 때, $w > 0$이고 $V_i < V_f$일 때 $w < 0$이다. 계가 비가역적인 일을 할 경우 얻을 수 있는 일의 양은 가역적일 때보다 더 적다.
            \par 일반적으로, $\difform U = \difform q + \difform w_{\mathrm{exp}} + \difform w_{\mathrm{add}}$로 나타낼 수 있다. 이때 
            계에 가해지는 일이 없을 때, $\difform U = \difform q$로 나타낼 수 있다. 만약 등적 과정일 때, 이는 다시 한 번 
            $\difform U=\difform q_V$로 나타낼 수 있다. 따라서 양변을 적분하면
                \begin{equation*}
                    \int_{i}^{f}\difform U = \int_{i}^{f}\difform q_V
                \end{equation*}
            즉 $\Delta U = q_V$로 나타낸다.
            \par \textbf{열량 측정법(Calorimetry)}은 열역학 과정에서 일어나는 열의 출입을 측정한다. 여기에는 \textbf{열량계(Calorimeter)}가 이용되고, 
            $q_V$를 측정하기 위해서는 \textbf{단열 봄 열량계(Adiabatic bomb calorimeter)}가 이용된다. 봄 열량계는 용량이 일정한 반응기('Bomb'), 열을 흡수할 수조, 
            온도계로 이루어진다. 이때 외부와의 열 출입은 없다고 가정한다. 열용량 $C$에 대해 출입한 열은 온도 변화 $\Delta T$에 대해 $q = C\Delta T$로 측정되고, 
            열량계의 열용량은 전기적으로 $q = It\Delta \phi$로 측정된다. 이때 $I$는 전류, $t$는 전류가 흐른 시간, $\Delta \phi$는 전위차이다. 또는 일정량의 알려진 
            물질(주로 벤조산을 이용)을 연소함으로써 측정된다.
            \par 온도-내부 에너지 그래프의 접선의 기울기를 그 온도에서의 \textbf{열용량(Heat capacity)}이라 한다. \textbf{등적 열용량(Heat capacity at constant volume)}은 
            기호로 $C_V$로 나타내고, 다음과 같이 정의된다:
            \begin{defn}[등적 열용량]
                \begin{equation*}
                    C_V = \left( \frac{\partial U}{\partial T}\right)_V
                \end{equation*}
            \end{defn}
            열용량은 크기 성질로, 이를 세기 성질로 변환하려면 몰수로 나누어 \textbf{등적 몰 열용량(Molar heat capacity at constant volume)}을 구한다. 
            기호로 $C_{V,m} = C_V / n$으로 나타낸다.
            \par \begin{fact}[편미분의 성질]
            (참고) 편미분은 다음과 같은 성질을 만족한다:
                \begin{enum}
                    \item $z$가 일정할 때, $\displaystyle\left( \frac{\partial f}{\partial x} \right)_z = \left(\frac{\partial f}{\partial x} \right)_y + \left(\frac{%
                    \partial f}{\partial y}\right)_x \left(\frac{\partial y}{\partial x} \right)_z$
                    \item $\displaystyle\left( \frac{\partial y}{\partial x} \right)_z = \displaystyle\frac{1}{\left( \displaystyle\frac{\partial x}{\partial y}\right)_z}$
                    \item \textbf{(Euler chain relation)} $\displaystyle\left(\frac{\partial x}{\partial y}\right)_z \left(\frac{\partial y}{\partial z}\right)_x \left( \frac{\partial z}{\partial x}\right)_y = -1$ 
                \end{enum}
            \end{fact}
            물질의 \textbf{비열(Specific heat capacity)}은 물질의 열용량을 질량으로 나눈 것으로, $C_{V,s} = C_V / m$로 정의된다.
            \par 내부 에너지는 다음과 같은 관계를 만족한다:
            \begin{obs}[내부 에너지와 등적 열용량의 관계]
                \begin{equation*}
                    \difform U = C_V \difform T
                \end{equation*}
            \end{obs}
            즉 양변을 적분하면 다음을 만족한다:
                \begin{equation*}
                    \Delta U = \int_{T_i}^{T_f}C_V \difform T = C_V \Delta T
                \end{equation*}
            따라서 $q_V = C_V \Delta T$를 만족한다.
    \section{엔탈피}\label{enthal}
            \hspace{\parindent} 화학 반응은 보통 일정 압력 조건에서 일어나므로, 계의 에너지를 이와 관련된 변수로 다시 정의한다. 이를 \textbf{엔탈피(Enthalpy, $H$)}라 하고, 
            다음과 같이 정의한다:
            \begin{defn}[엔탈피]
                \begin{equation*}
                    H = U + pV
                \end{equation*}
            \end{defn}
            이때 $p$는 계의 압력, $V$는 계의 부피, $U$는 계의 내부 에너지이다. $U$, $p$, $V$가 모두 상태 함수이므로 엔탈피 또한 상태 함수이다. 엔탈피 변화 $\difform H$는 
            다음과 같이 나타낼 수 있다:
                \begin{equation*}
                    \begin{aligned}
                        \difform H &= \difform \left(U+pV\right)\\
                        &= \difform U + p \difform V + V \difform p\\
                        &= \difform q + \difform w + p \difform V + V \difform p\\
                        &= \difform q + \cancel{\left(-p \difform V + p \difform V\right)} + V \difform p\\
                        &= \difform q + V \difform p
                    \end{aligned}
                \end{equation*}
            이때 압력이 일정하면 $\difform p = 0$이므로, 다음이 성립한다:
                \begin{equation*}
                    \difform H = \difform q_p
                \end{equation*}
            즉, 양변을 적분하면 다음이 성립한다:
            \begin{obs}[엔탈피 변화와 등압 열의 관계]
                \begin{equation*}
                    \Delta H = q_p
                \end{equation*}
            \end{obs}
            \par 엔탈피를 측정하기 위해서는 \textbf{등압 열량계(Isobaric calorimeter)}를 이용한다. 연소 반응에 대해서는 \textbf{단열 불꽃 열량계(Adiabatic flame calorimeter)}를 
            이용하기도 한다. 일반적으로는 \textbf{주사 시차 열량계(Differential Scanning Calorimeter, DSC)}를 이용한다. 이에 대해서는 \ref{thermalchem}에서 더 자세히 설명할 것이다. 또한 
            내부 에너지와 엔탈피는 열량계로써 측정할 필요는 없고, 이는 \ref{echemcell}에서 다시 다룰 것이다.
            \par 또한 기체 상태에서 엔탈피는 이상 기체 상태방정식을 이용하여 다음과 같이 나타낼 수 있다:
                \begin{equation*}
                    \Delta H = \Delta U + \Delta n_g RT
                \end{equation*}
            이때 $\Delta n_g$는 기체 분자의 몰수 변화이다. 이를 통해 엔탈피 변화와 내부 에너지 변화 사이의 차이를 구할 수 있다.
            \par \textbf{등압 열용량(Heat capacity at constant pressure}, 또는 isobaric heat capacity\textbf{)}은 다음과 같이 정의된다:
            \begin{defn}[등압 열용량]
                \begin{equation*}
                    C_p = \left(\frac{\partial H}{\partial T}\right)_p
                \end{equation*}
            \end{defn}
            또한, \textbf{등압 몰 열용량(Molar heat capacity at constant pressure, $C_{p,m}$)}은 등압 열용량을 물질의 몰수로 나눈 것이다. 따라서 다음이 성립한다:
                \begin{equation*}
                    \difform H = C_p \difform T
                \end{equation*}
            이때 양변을 적분하면
                \begin{equation*}
                    \Delta H = C_p \Delta T
                \end{equation*}
            이 성립하고, 따라서 $q_p = C_p \Delta T$가 성립한다. $C_{p,m}$은 다음과 같이 근사할 수 있다:
                \begin{equation*}
                    C_{p,m} = a + bT + \frac{c}{T^2}
                \end{equation*}
            이때 $a$, $b$, $c$는 실험적으로 구한다.
            \par 이상 기체에서 다음이 성립한다:
            \begin{obs}[등압 및 등적 열용량 사이의 관계]
                \begin{equation*}
                    C_p - C_V = nR
                \end{equation*}
            \end{obs}
            이는 \ref{exactdiff}에서 증명할 것이다.
    \section{열화학} \label{thermalchem}
            \hspace{\parindent} 화학 반응 도중 출입하는 열에 대한 논의를 \textbf{열화학(Thermochemistry)}이라 한다.
            \begin{defn}[발열 반응과 흡열 반응]
            화학 반응에서 $\Delta H < 0$인 반응을 \textbf{발열(Exothermic)} 반응이라 하고, 
            $\Delta H > 0$인 반응을 \textbf{흡열(Endothermic)} 반응이라 한다.\footnote[7]{%
            더 정확히는 $\Delta H < 0$일 때 \textbf{Exenthalpic}, $\Delta H > 0$일 때 \textbf{Endenthalpic}이라 한다.}
            \end{defn}
            \par 엔탈피 변화는 표준 조건에서 일어나는 반응을 기준으로 표현한다.
            \begin{defn}[표준 상태]
            어떤 물질의 특정 온도에서의 \textbf{표준 상태(Standard state)}는 그 온도에서 $1$ bar의 압력을 가질 때 순수한 물질의 상태를 의미한다. 
            \end{defn}
            \begin{defn}[표준 엔탈피 변화]
            \textbf{표준 엔탈피 변화(Standard enthalpy change, $\Delta H^{\circlehbar}$)}는 반응물과 생성물이 표준 상태에 있을 때의 
            엔탈피 변화를 일컫는다.
            \end{defn}
            예를 들어, $298$ K에서 액체 에탄올의 표준 상태는 $298$ K, $1$ bar에서 순수한 액체 에탄올을 의미한다. 
            \par 표준 엔탈피의 종류에 따라 여러 이름을 붙일 수 있다: 예를 들어 물의 $373$ K에서의 표준 기화 엔탈피 
            $\Delta_{\mathrm{vap}}H^{\circlehbar}\left(373 \mathrm{K}\right) = +40.66 \mathrm{kJ mol}^{-1}$이다. 
            관습적으로 열역학 자료를 표현할 때에는 $298.15$ K일 때의 자료를 표시하거나, 온도를 명시한다.
            \par 반응 엔탈피에는 다음과 같은 것들이 있다:
            \begin{table}[H]
            \centering
                \begin{tabular}{ c c c }
                    \hline
                    \rowcolor{lightgray}
                    변화 & 반응 & 기호 \\
                    \hline
                    상태 변화 (Transition) & $\alpha$상 $\rightarrow$ $\beta$상 & $\Delta_{\mathrm{trs}}H$\\
                    융해 (Fusion) & s $\rightarrow$ l & $\Delta_{\mathrm{fus}}H$ \\
                    기화 (Vaporization) & l $\rightarrow$ g & $\Delta_{\mathrm{vap}}H$ \\
                    승화 (Sublimation) & s $\rightarrow$ g & $\Delta_{\mathrm{sub}}H$ \\
                    혼합 (Mixing) & 순물질 $\rightarrow$ 혼합물 & $\Delta_{\mathrm{mix}}H$ \\
                    용해 (Solution) & 용질 $\rightarrow$ 용액 & $\Delta_{\mathrm{sol}}H$ \\
                    수화 (Hydration) & $X^{\pm}\left(\mathrm{g}\right) \rightarrow X^{\pm}\left(\mathrm{aq}\right)$ & $\Delta_{\mathrm{hyd}}H$ \\
                    원자화 (Atomization) & 화학종(s, l, g) $\rightarrow$ 원자(g) & $\Delta_{\mathrm{at}}H$ \\
                    이온화 (Ionization) & $X \left(g\right) \rightarrow X^{+}\left(g\right)+ e^{-}\left(g\right)$ & $\Delta_{\mathrm{ion}}H$ \\
                    전자 포획 (Electron gain) & $X \left(g\right) + e^{-}\left(g\right)\rightarrow X^{-}\left(g\right)$ & $\Delta_{\mathrm{eg}}H$ \\
                    반응 (Reaction) & 반응물 $\rightarrow$ 생성물 & $\Delta_{\mathrm{r}}H$ \\
                    연소 (Combustion) & 화학종(s, l, g) $+ \mathrm{O}_2 \left(g\right) \rightarrow$ 산화물(s, l, g) & $\Delta_{\mathrm{c}}H$ \\
                    생성 (Formation) & 원소 $\rightarrow$ 화합물 & $\Delta_{\mathrm{f}}H$ \\
                    활성화 (Activation) & 반응물 $\rightarrow$ 활성화물 & $\Delta^{\ddagger}H$ \\
                    \hline
                \end{tabular}
            \end{table}
            \par 엔탈피는 상태함수이므로, 얼음이 수증기로 승화되는 반응의 표준 승화 엔탈피 $\Delta_{\mathrm{sub}}H^{\circlehbar}$는 얼음이 물로 융해되는 반응의 
            표준 융해 엔탈피 $\Delta_{\mathrm{fus}}H^{\circlehbar}$과 물이 수증기로 기화되는 반응의 표준 기화 엔탈피 $\Delta_{\mathrm{vap}}H^{\circlehbar}$의 
            합과 같다. 즉 $\Delta_{\mathrm{sub}}H^{\circlehbar} = \Delta_{\mathrm{fus}}H^{\circlehbar} + \Delta_{\mathrm{vap}}H^{\circlehbar}$를 만족하고, 
            이를 그림으로 나타내면 Figure \ref{f1}과 같다. 또한 역반응의 표준 엔탈피는 정반응의 표준 엔탈피와 부호가 다르고 크기는 같다.
            \begin{figure}[H]
                \centering
                \begin{tikzpicture}
                    \draw[gray, thick, ->] (-3,-2.5) -- (-3,2) node[rotate = 90, midway, above] {엔탈피, $H$};
                    \draw[gray, thick] (-2,-2) -- (2,-2) node[right] {s};
                    \draw[gray, thick] (-2,-0.5) -- (0,-0.5) node[right] {l};
                    \draw[gray, thick] (-2, 1.5) -- (2, 1.5) node[right] {g};
                    \draw[black, thick, ->] (-1.5,-2) -- (-1.5,-0.5) node[midway, right] {$\Delta_{\mathrm{fus}}H^{\circlehbar}$};
                    \draw[black, thick, ->] (-1.5,-0.5) -- (-1.5,1.5) node[midway, right] {$\Delta_{\mathrm{vap}}H^{\circlehbar}$};
                    \draw[black, thick, ->] (1.5,-2) -- (1.5,1.5) node[midway, right] {$\Delta_{\mathrm{sub}}H^{\circlehbar}$};
                \end{tikzpicture}
                \caption{상태 변화의 Born-Haber 순환}\label{f1}
            \end{figure}
            \par 반응 엔탈피 $\Delta_{\mathrm{r}}H^{\circlehbar}$는 다음과 같이 정의된다:
            \begin{defn}[반응 엔탈피]
            \begin{equation*}
                \Delta_{\mathrm{r}}H^{\circlehbar} = \sum_{\textrm{생성물}} \nu H^{\circlehbar}_{m} - \sum_{\textrm{반응물}} \nu H^{\circlehbar}_{m}
            \end{equation*}
            이때 $\nu$는 반응식의 화학종별 계수이다.
            \end{defn}
            \par 표준 반응 엔탈피는 다른 반응의 반응 엔탈피를 구하기 위해 이용될 수 있다. 이때 이용되는 법칙은 \textbf{Hess의 법칙(Hess's law)}이다:
            \begin{law}[Hess의 법칙]
                어떤 반응의 표준 반응 엔탈피는 그 반응을 나누었을 때 각 반응의 표준 반응 엔탈피의 합과 같다.
            \end{law}
            \vspace{2pt}
            \par 이때 주로 이용되는 것은 \textbf{표준 생성 엔탈피(Standard enthalpy of formation, $\Delta_\mathrm{f} H^{\circlehbar}$)}로, 
            어떤 화학종이 \textbf{표준 상태(Reference state)}의 원소로부터 생성될 때의 반응 엔탈피를 말한다. 표준 상태는 다음과 같이 정의된다:
            \begin{defn}[표준 상태]
                어떤 원소의 표준 상태는 주어진 온도와 $1$ bar에서 가장 안정한 상태를 말한다.
            \end{defn}
            \vspace{2pt}
            실제로 반응 엔탈피의 계산은 다음과 같이 이루어진다:
            \begin{equation*}
                \Delta_\mathrm{r}H^{\circlehbar} = \sum_{\textrm{생성물}}\nu \Delta_{\mathrm{f}}H^{\circlehbar} - \sum_{\textrm{반응물}}\nu \Delta_{%
                \mathrm{f}}H^{\circlehbar}
            \end{equation*}
            또한 Hess의 법칙에 따라 다음 Figure \ref{f2}가 성립한다.
            \begin{figure}[H]
                \centering
                \begin{tikzpicture}
                    \draw[gray, thick, ->] (-3,-2.5) -- (-3,2) node[rotate = 90, midway, above] {엔탈피, $H$};
                    \draw[gray, thick] (-2,-2) -- (2,-2) node[midway, above] {생성물};
                    \draw[gray, thick] (-2,-0.5) -- (0,-0.5) node[right, above] {반응물};
                    \draw[gray, thick] (-2, 1.5) -- (2, 1.5) node[right] {원소};
                    \draw[black, thick, ->] (-1.5,-0.5) -- (-1.5,-2) node[midway, right] {$\Delta_{\mathrm{r}}H^{\circlehbar}$};
                    \draw[black, thick, ->] (-1.5,-0.5) -- (-1.5,1.5) node[midway, right] {$-\Delta_{\mathrm{f}}H^{\circlehbar}\left(\textrm{반응물}\right)$};
                    \draw[black, thick, ->] (1.5,1.5) -- (1.5,-2) node[midway, right] {$\Delta_{\mathrm{f}}H^{\circlehbar}\left(\textrm{생성물}\right)$};
                \end{tikzpicture}
                \caption{반응 엔탈피와 생성 엔탈피 간의 관계}\label{f2}
            \end{figure}
        \par 이때 화학종 $J$에 대해 \textbf{화학량수(Stoichiometric number, $\nu_\mathrm{J}$)}를 도입하면 
        반응물에서는 음수, 생성물에서는 양수이므로 다음이 성립한다:
        \begin{defn}
        \begin{equation*}
            \Delta_{\mathrm{r}}H^{\circlehbar} = \sum_{\mathrm{J}}\nu_{\mathrm{J}}\Delta_{\mathrm{f}}H^{\circlehbar}\left(\mathrm{J}\right)
        \end{equation*}
        \end{defn}
        \par 순물질과 마찬가지로 반응에서도 엔탈피 변화와 등적 열용량 사이의 관계가 존재한다. 이를 \textbf{Kirchhoff의 법칙(Kirchhoff's law)}이라 한다:
        \begin{law}[Kirchhoff의 법칙]\label{kirlaw}
        \begin{equation*}
            \Delta_\mathrm{r}H^{\circlehbar}\left(T_2 \right) = \Delta_\mathrm{r}H^{\circlehbar}\left( T_1 \right) + \int_{T_1}^{T_2}\Delta_\mathrm{r}C^{\circlehbar}_{p}\difform T
        \end{equation*}
        \end{law}
        이때 $\displaystyle\Delta_{\mathrm{r}}C^{\circlehbar}_{p} = \sum_{\mathrm{J}}\nu_{\mathrm{J}}C^{\circlehbar}_{p,m}\left(\mathrm{J}\right)$를 만족하고, 따라서 다음을 만족한다: 
        \begin{equation*}
            \Delta_\mathrm{r}H^{\circlehbar}\left(T_2 \right) = \Delta_\mathrm{r}H^{\circlehbar}\left(T_1 \right) + \Delta_\mathrm{r}C^{\circlehbar}_{p}\left(T_2 - T_1 \right)
        \end{equation*}
        \par 실험적으로는 \textbf{DSC}와 \textbf{등온 적정 열량 측정법(Isothermal titration calorimetry, ITC)}이 있다.
        \par DSC에서는 단위 시간 동안 일정한 온도 변화 $\alpha$(K s$^{-1}$)를 주어, 출입하는 열량을 측정한다. 만약 어떠한 물리적·화학적 변화도 일어나지 않는다고 
        가정하면(Reference), 시간 $t$에서 $q_p = C_p \alpha t$의 열량이 출입한다. 그러나 물리적·화학적 변화가 있을 경우(Sample), 이때 온도를 Reference와 같게 하기 
        위해 추가적으로 출입한 열량을 $q_{p,\mathrm{ex}}$라 하면 총 출입한 열량은 $q_p + q_{p,\mathrm{ex}}$이다. 따라서 물질 자체의 변화에서 
        $C_{p,\mathrm{ex}}$를 측정하면 다음과 같은 식을 만족한다:
        \begin{equation*}
            C_{p,\mathrm{ex}} = \frac{q_{p,\mathrm{ex}}}{\Delta T} = \frac{q_{p,\mathrm{ex}}}{\alpha t} = \frac{P_{\mathrm{ex}}}{\alpha}
        \end{equation*}
        이때 $P_{\mathrm{ex}}$는 온도를 같게 맞추기 위해 더 출입한 전력이다. 따라서 $x$축을 온도, $y$축을 $C_{p,\mathrm{ex}}$로 하는 
        \textbf{시차 열분석 곡선(서모그램, Thermogram)}을 얻을 수 있다. 따라서 물질의 상태 변화에 출입한 엔탈피는 다음과 같이 구할 수 있다:
        \begin{equation*}
            \Delta H = \int_{T_1}^{T_2}C_{p,\mathrm{ex}}\difform T
        \end{equation*}
        \par 마찬가지로 ITC에서는 적정 시 출입하는 열량을 온도를 같게 맞추기 위한 전력 출입으로 변환하여 측정한다. ITC 그래프를 적분하면 적정 과정 전체의 엔탈피 
        변화 그래프를 얻을 수 있다.
    \section{상태 함수와 완전미분}\label{exactdiff}
        \hspace{\parindent} \textbf{완전미분(Exact differential)}은 상태 함수의 전미분을 말하고, \textbf{불완전미분(Inexact differential)}은 
        경로 함수의 전미분을 말한다. 예를 들어, 상태 함수인 내부 에너지의 전미분 $\difform U$는 완전미분이고, 경로 함수인 열의 전미분 $\difform q$이나 
        일의 전미분 $\difform w$은 불완전미분이다. 불완전미분일 때, $\difform $나 $\delta$ 위에 줄을 긋기도 한다.
        \par 내부 에너지는 다음과 같이 나타낼 수 있다:
        \begin{cor}[내부 에너지의 전미분]
        \begin{equation*}
            \difform U = \left(\frac{\partial U}{\partial V}\right)_T \difform V + \left(\frac{\partial U}{\partial T}\right)_V \difform T
        \end{equation*}
        \end{cor}
        이때 $\displaystyle\left(\frac{\partial U}{\partial T}\right)_V = C_V$가 됨은 앞의 \ref{intener}에서 살펴보았다. 또한, \textbf{내부 압력(Internal pressure)} $\pi_T$를 
        다음과 같이 정의하자:
        \begin{defn}[내부 압력]
        \begin{equation*}
            \pi_T = \left( \frac{\partial U}{\partial V}\right)_T
        \end{equation*}
        \end{defn}
        따라서 내부 에너지의 전미분은 다음과 같이 나타낼 수 있다:
        \begin{cor}[내부 에너지의 전미분]
        \begin{equation*}
            \difform U = \pi_T \difform V + C_V \difform T
        \end{equation*}
        \end{cor}
        이때 분자 사이에 어떠한 상호작용도 없다면 $\pi_T = 0$이다.
        \par 압력이 일정할 때, $\displaystyle\left( \frac{\partial U}{\partial T}\right)_p$를 구하면 다음과 같다:
        \begin{equation*}
            \left(\frac{\partial U}{\partial T}\right)_p = \pi_T\left(\frac{\partial V}{\partial T}\right)_p + C_V
        \end{equation*}
        이때 압력이 일정할 때 온도에 따른 부피의 변화량, 즉 \textbf{열팽창률(Expansion coefficient)} $\alpha$는 다음과 같이 정의된다:
        \begin{defn}[열팽창률]
        \begin{equation*}
            \alpha = \frac{1}{V} \left(\frac{\partial V}{\partial T} \right)_p
        \end{equation*}
        \end{defn}
        또한 \textbf{등온 압축률(Isothermal compressibility)} $\kappa_T$를 다음과 같이 정의한다:
        \begin{defn}[등온 압축률]
        \begin{equation*}
            \kappa_T = -\frac{1}{V}\left(\frac{\partial V}{\partial p}\right)_T
        \end{equation*}
        \end{defn}
        이는 온도가 일정할 때 압력이 증가함에 따른 부피의 감소량을 의미한다.
        \par 따라서 $\displaystyle\left(\frac{\partial U}{\partial T}\right)_p = \alpha \pi_T V + C_V$를 만족하고, 이상 기체에서는 $\pi_T = 0$이므로 
        $\displaystyle\left(\frac{\partial U}{\partial T}\right)_p = C_V$를 만족한다.
        \par $C_p$와 $C_V$의 정의에 따라 이상 기체에서 이 둘의 차이는 다음을 만족한다:
        \begin{cor}[등압 열용량과 등적 열용량의 차, 이상 기체에서]
        \begin{equation*}
            \begin{aligned}
                C_p - C_V &= \left(\frac{\partial H}{\partial T}\right)_p - \left(\frac{\partial U}{\partial T} \right)_p\\
                &= \left(\frac{\partial \left(U-nRT\right)}{\partial T}\right)_p - \left(\frac{\partial U}{\partial T}\right)_p\\
                &= nR
            \end{aligned}
        \end{equation*}
        \end{cor}
        일반적으로는 다음을 만족한다:
        \begin{cor}[등압 열용량과 등적 열용량의 차, 일반화]\label{cpcvgen}
        \begin{equation*}
            C_p - C_V = \frac{\alpha^2 TV}{\kappa_T}
        \end{equation*}
        \end{cor}
        이상 기체에서, $\alpha = 1/T$이고 $\kappa_T = 1/p$이므로 \ref{cpcvgen}을 만족한다.
        \par 엔탈피에서도 마찬가지로,
        \begin{cor}[엔탈피의 전미분]
        \begin{equation*}
            \difform H = \left(\frac{\partial H}{\partial p}\right)_T \difform p + \left(\frac{\partial H}{\partial T}\right)_p \difform T
        \end{equation*}
        \end{cor}
        를 만족한다. 이때 $\displaystyle\left(\frac{\partial H}{\partial T}\right)_p = C_p$이고, 
        $\displaystyle\left(\frac{\partial H}{\partial p}\right)_T = -\left(\frac{\partial T}{\partial p}\right)_H \left(\frac{\partial H}{\partial T}\right)_p$이므로 
        \textbf{Joule-Thomson 계수($\mu$)}를 $\displaystyle\mu = \left(\frac{\partial T}{\partial p}\right)_H$로 정의하면 다음이 성립한다:
        \begin{equation*}
            \difform H = -\mu C_p \difform p + C_p \difform T
        \end{equation*}
        \par 이때 \textbf{등엔탈피(Isenthalpic)} 조건에서 부피가 변화하면 온도가 변화하는 것을 \textbf{Joule-Thomsom 효과(Joule-Thomson effect)}라 한다. 
        이러한 효과는 실제 기체에서만 관찰된다; 이상 기체에서는 $\mu = 0$이기 때문이다. 실제 기체의 Joule-Thomson 계수는 다음 Figure \ref{f3}과 같이 관찰된다. 
        \begin{figure}[H]
            \centering
            \begin{tikzpicture}
                \filldraw[color=gray, fill=gray!10, very thick] (-5,-2) .. controls (2,-1.5) and (2,0.1) .. (-5,1.7);
                \draw[black, very thick, ->] (-5,-3) -- (-5,3) node[rotate = 90, midway, above] {온도, $T$};
                \draw[black, very thick, ->] (-5,-3) -- (5,-3) node[midway, below] {압력, $p$};
                \node[] at (-4,0.3) {$\mu > 0$};
                \node[] at (-4, -0.3) {냉각};
                \node[] at (2,0.3) {$\mu < 0$};
                \node[] at (2, -0.3) {가열}; 
            \end{tikzpicture}
            \caption{실제 기체의 온도-압력 그래프와 Joule-Thomson 계수}\label{f3}
        \end{figure}
        \par 이때 같은 압력에서 $\mu < 0$에서 $\mu > 0$으로 바뀌는 온도를 하부(Lower) \textbf{역전 온도(Inversion temperature)}, $\mu > 0$에서 $\mu < 0$으로 바뀌는 온도를 
        상부(Upper) 역전 온도라고 한다. 이를 분자 운동론적으로 해석하면, 등엔탈피 조건에서 분자 간 인력이 주로 작용할 때($Z<1$) 압력을 높이면 온도가 상승한다($\mu>0$). 
        반대로, 분자 간 반발력이 주로 작용할 때($Z>1$) 압력을 높이면 온도가 하강한다($\mu<0$).
    \section{단열 과정}
        \hspace{\parindent}단열 과정에서는 계의 열 출입이 없다. 따라서 $\difform q = 0$을 만족하므로, $\Delta U = \difform w_{\mathrm{ad}}$를 만족한다. 
        한편, \ref{intener}에서 $\Delta U = C_V \Delta T$임을 보였다. 따라서 다음이 성립한다:
        \begin{equation*}
            w_{\mathrm{ad}} = C_V \Delta T
        \end{equation*}
        따라서 가역 과정에서 $w = -p \difform V$이므로, 이상 기체에서 다음 과정이 성립한다:
        \begin{equation*}
        \begin{aligned}
            C_V \difform T &= -p\difform T = -\frac{nRT\difform T}{V}\\
            \frac{C_V \difform T}{T} &= -\frac{nR\difform V}{V}\\
            C_V \int_{T_i}^{T_f} \frac{\difform T}{T} &= -nR \int_{V_i}^{V_f} \frac{\difform V}{V}\\
            C_V \ln{\frac{T_f}{T_i}} &= -nR\ln{\frac{V_f}{V_i}}\\
            \frac{C_V}{nR} \ln{\frac{T_f}{T_i}} &= \ln{\frac{V_i}{V_f}}
        \end{aligned}
        \end{equation*}
        이때 $C_V /nR = C_{V,m}/R$이고 이 값을 $c$로 두면
        \begin{equation*}
            \begin{aligned}
                c \ln{\frac{T_f}{T_i}} &= \ln{\frac{V_i}{V_f}}\\
                \ln{\left(\frac{T_f}{T_i}\right)^c} &= \ln{\frac{V_i}{V_f}}
            \end{aligned}
        \end{equation*}
        가 성립한다. 따라서 양변을 정리하면 다음이 성립한다:
        \begin{obs}[단열 과정에서 부피와 온도 사이의 관계]
        \begin{equation*}
            V_i {T_i}^c = V_f {T_f}^c,\ c=\frac{C_{V,m}}{R}
        \end{equation*}
        \end{obs}
        \par 마찬가지로, 이상 기체에서 $C_{p,m}-C_{V,m} = R$이 성립하므로, 단열 과정에서 다음이 성립한다:
        \begin{equation*}
            \begin{aligned}
                \frac{p_i V_i}{p_f V_f} = \frac{T_i}{T_f} &= \left(\frac{V_f}{V_i}\right)^{1/c}\\
                \frac{p_i}{p_f}\left(\frac{V_i}{V_f}\right)^{\frac{1}{c} + 1} &= 1
            \end{aligned}
        \end{equation*}
        이때 $\displaystyle\frac{1}{c} + 1 = \frac{R + C_{V,m}}{C_{V,m}} = \frac{C_{p,m}}{C_{V,m}} = \gamma$라 두면,
        \begin{equation*}
            \frac{p_i}{p_f}\left(\frac{V_i}{V_f}\right)^{\gamma} = 1
        \end{equation*}
        이다. 즉 다음이 성립한다:
        \begin{cor}[단열 과정에서 압력과 부피 사이의 관계]
        \begin{equation*}
            p_f {V_f}^{\gamma} = p_i {V_i}^{\gamma}
        \end{equation*}
        \end{cor}
        \par 단원자 이상 기체에서, $\displaystyle C_{V,m} = \frac{3}{2}R$이고 $\displaystyle C_{p,m} = \frac{5}{2}R$이므로 \textbf{$\displaystyle\gamma = \frac{5}{3}$}이다. 다원자 이상 기체에서는 
        분자의 모양에 따라 다른 값을 취한다. $\gamma >1$이므로, 단열 과정($\displaystyle p \propto 1/{V^{\gamma}}$)에서는 등온 과정($\displaystyle p \propto 1/V$)에서보다 더 가파른 기울기를 보인다.