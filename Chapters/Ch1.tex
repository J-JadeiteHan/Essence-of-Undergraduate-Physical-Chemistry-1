\chapter{기체의 성질}
        \section{이상 기체}
            \hspace{\parindent} 기체의 상태를 나타내는 변수에는 압력, 온도, 부피, 몰수가 있다. 각 변수에 대응되는 SI 단위는 \textrm{Pa(= N m$^{-2}$), K, m$^3$, mol}이다.
            \begin{defn}[표준 압력]
            $1$ \textrm{bar}(= $10^5$ \textrm{Pa})의 압력을 \textbf{표준 압력(Standard pressure)}이라 하고, 기호로 $p^{\circlehbar}$으로 나타낸다.
            \end{defn}
            $0$ \degree\textrm{C}는 273.15 \textrm{K}로 정의된다. 상태방정식은 일반적으로 다음과 같이 표현된다:
                \begin{defn}[상태방정식]
                    $p = f \left( T,V,n \right)$
                \end{defn}
            경험적인 법칙으로부터 다음과 같은 법칙들을 유도할 수 있다:
                \begin{obs}[Boyle의 법칙]
                    $pV = \text{일정, 단 }n \text{과 }T \text{가 일정}$
                \end{obs}
                \begin{obs}[Charles의 법칙(1)]
                    $V = \text{상수} \times T \text{, 단 }n \text{과 }p \text{가 일정}$
                \end{obs}
                \begin{obs}[Charles의 법칙(2)]
                    $p = \text{상수} \times T \text{, 단 }n \text{과 }V \text{가 일정}$
                \end{obs}
                \begin{obs}[Boyle-Charles 법칙]
                    $\displaystyle\frac{pV}{T} = \text{일정, 단 }n \text{이 일정}$
                \end{obs}
                \begin{obs}[Avogadro의 법칙]
                    $V = \text{상수} \times n \text{ , 단 }p \text{와 } T \text{가 일정}$
                \end{obs}
            같은 온도를 나타낸 곡선을 \textbf{등온 곡선(Isotherm)}, 같은 압력을 나타낸 곡선을 \textbf{등압 곡선(Isobar)}, 
            같은 부피를 나타낸 곡선을 \textbf{등적 곡선(Isochore)}이라고 한다.\\
            따라서 다음과 같은 \textbf{이상 기체 상태방정식}을 유도할 수 있다:
                \begin{thm}[이상 기체 상태방정식]
                    $pV = nRT$
                \end{thm}
            이때 $R$을 \textbf{기체 상수(Gas constant)}라 하고, 그 값은 $8.314\,4621$ \textrm{J K$^{-1}$ mol$^{-1}$}이다. 
            실제 기체는 $p \rightarrow 0$일 때 이상 기체와 유사하게 거동한다.
            \par 물질의 상태를 나타낼 때 온도와 압력을 표시해야 할 때가 많다. 대표적으로 다음과 같은 것들이 있다.
            \begin{defn}[SATP]
                $298.15$ \textrm{K}, $1$ \textrm{bar}를 Standard Ambient Temperature and Pressure, \textbf{SATP}라 한다.
            \end{defn}
            \begin{defn}[STP]
            $273.15$ \textrm{K}, $1$ \textrm{atm}를 Standard Temperature and Pressure, \textbf{STP}라 한다.
            \end{defn}
            STP에서 기체 $1$ \textrm{mol}의 부피는 
            $22.414$ \textrm{dm}$^3$에 해당한다.
            \begin{defn}[몰 부피]
            기체 1 \textrm{mol}의 부피를 $V_m$으로 나타내고, $V_m = V/n$을 만족한다.
            \end{defn}
            \par 혼합 기체에서는, 기체 각 성분의 \textbf{부분 압력(Partial pressure)}을 고려한다. 성분 $J$의 부분 압력 $p_J$는 다음과 같이 정의된다.
            \begin{defn}[부분 압력]
                $p_J = x_J p$
            \end{defn}
            이때 $x_J$는 성분 $J$의 \textbf{몰 분율(Mole fraction)}이고, 총 몰수 $n$에 대해 다음과 같이 정의된다.
                \begin{defn}[몰 분율]
                    $x_J = \displaystyle\frac{n_J}{n}, n = n_A + n_B + \cdots$
                \end{defn}
            $J$가 존재하지 않을 때에는 $x_J = 0$이고, $J$만 존재할 때에는 $x_J = 1$이다. 
            \begin{cor}
                모든 성분에 대해 $\displaystyle\sum_{J} x_J = 1$을 만족한다.
            \end{cor} 
            따라서 다음을 만족한다.
                \begin{cor}
                    $\displaystyle\sum_{J} p_J = \left( \sum_{J} x_J \right) p = p$
                \end{cor}
            이 관계는 이상 기체와 실제 기체 모두에서 적용된다. 이는 \textbf{Dalton 법칙(Dalton's law)}으로부터 유도된다.
            \begin{law}[Dalton 법칙]
            혼합 기체의 압력은 각 성분이 용기를 홀로 차지할 때 작용하는 압력의 합과 같다.
            \end{law}
            \begin{warn}
            \textbf{이 관계는 이상 기체에만 적용된다.} 그러나 \textbf{부분 압력은 모든 기체에 적용 가능하다.}
            \end{warn}
        \section{기체 분자 운동론}
            \hspace{\parindent} 기체 분자 운동론을 통해 이상 기체 상태방정식을 유도해 보자.
            \begin{rem}[기체 분자 운동론의 가정]
            기체 분자 운동론은 다음의 세 가지 가정으로부터 출발한다.
                \begin{enum}
                    \item 기체는 고전 역학을 따르는, 끊임없는 무작위 운동을 한다.
                    \item 분자의 크기는 무시 가능하다: 충돌과 충돌 간 기체 분자가 움직이는 거리보다 기체 분자 자체의 크기는 매우 작아서, 분자를 점으로 취급 가능하다.
                    \item 분자 간에는 완전 탄성 충돌만으로 상호작용한다.
                \end{enum}
            \end{rem}
            \begin{proof}
            운동량 보존에 의해, 분자는 충돌 전 $x$방향으로 $mv_x$의 운동량을 가지고, $x$축에 수직인 벽과 충돌 후에는 $-mv_x$의 운동량을 가진다. $\Delta t$의 
            시간 후에 이 분자가 벽과 충돌한다고 가정했을 때, 벽과의 거리가 $v_x \Delta t$ 이내이고 벽을 향해 운동하는 분자는 모두 벽과 충돌한다. 
            벽의 면적이 $A$라 할 때, 이 공간의 부피는 $Av_x \Delta t$이다.
            \par 입자의 몰수가 $n$, 용기의 부피가 $V$일 때 기체의 
            개수-밀도(number density)는 $\displaystyle\frac{n N_A}{V}$ ($N_A$는 아보가드로수)이므로, $Av_x \Delta t$의 부피 내에 있는 기체 분자 수는 
            $\displaystyle\frac{n N_A A v_x \Delta t}{V}$이다. 그러나 분자가 벽을 향해 운동할 확률은 $\displaystyle\frac{1}{2}$이고, 한편 운동량의 변화량은 $2mv_x$이므로, 
            운동량의 변화 $\Delta p_x$는 다음과 같다($M$은 분자량):
                \begin{equation*}
                    \begin{aligned}
                        \Delta p_x &= \frac{nN_A A v_x \Delta t}{2V} \times 2 m v_x \\
                        &= \frac{nmN_A A {v_x}^2 \Delta t}{V} = \frac{nMA {v_x}^2 \Delta t}{V}
                    \end{aligned}
                \end{equation*}
            따라서 힘 $F = \frac{dp}{dx}$에서, $x$축과 수직한 벽면에 가해지는 힘은 $F_x = \displaystyle\frac{\Delta p_x}{\Delta t} = \frac{nMA {v_x}^2}{V}$이다. \\
            이를 면적 $A$로 나누면, $x$축과 수직한 벽면에 가해지는 압력이 되고, 평균을 $\left< \cdot \right>$로 나타낼 때 평균 압력 $p$는 
                \begin{equation*}
                    p = \frac{nM \left< {v_x}^2 \right>}{V}
                \end{equation*}
            이다. $\left< {v_x}^2 \right> = \left< {v_y}^2 \right> = \left< {v_z}^2 \right>$이고, $v^2 = {v_x}^2 + {v_y}^2 + {v_z}^2 = 3 {v_x}^2$에서, 
            $\left< {v_x}^2 \right> = \frac{1}{3} \left< v^2 \right>$를 만족한다. 따라서 $v_{\mathrm{rms}} = \sqrt{\left\langle v^2 \right\rangle}$에서, 
            다음을 얻는다:
            \begin{obs}\label{pvconst}
                \begin{equation*}
                    pV = \frac{1}{3} n M {v_\mathrm{rms}}^2
                \end{equation*}
            \end{obs}
            따라서 $pV =$ 상수 를 만족한다.
            \end{proof}
            이는 Boyle의 법칙과 일치하는 결과이다.
            \par \textbf{Maxwell-Boltzmann 분포}는 다음과 같이 유도된다.
            \begin{obs}[Maxwell-Boltzmann 분포]
                Boltzmann 분포에 의해, 속도의 분포는 $\displaystyle f\left( v \right) = K e^{-\eps / kT}$로 나타낼 수 있다. 이때 입자는 운동 에너지만을 갖는다고 가정하였으므로,
                \begin{equation*}
                    \eps = \frac{1}{2} m {v_x}^2 + \frac{1}{2} m {v_y}^2 + \frac{1}{2} m {v_z}^2
                \end{equation*}
            을 만족한다. 따라서
                \begin{equation*}
                    \displaystyle f(v) = K e^{-\left(m{v_x}^2 + m{v_y}^2 + m{v_z}^2 \right)/2kT} = K e^{-m{v_x}^2 / 2kT} e^{-m{v_y}^2 / 2kT} e^{-m{v_z}^2 /2kT}
                \end{equation*}
            을 만족하고, $f \left( v \right) = f \left( v_x \right) f \left( v_y \right) f \left( v_z \right)$와 $K = K_x K_y K_z$라 하고 $x$ 방향을 구한다.\\
            $- \infty < v_x < \infty$를 만족하고, $\displaystyle\int_{-\infty}^{\infty} f \left( v_x \right) \difform v_x = 1$을 만족해야 하므로,
            \renewcommand*{\thefootnote}{\fnsymbol{footnote}}
                \begin{equation*}
                    1 = K_x \int_{- \infty}^{\infty} e^{-m {v_x}^2 / 2kT} \difform v_x = K_x {\left( \frac{2 \pi k T}{m} \right)}^{1/2}\footnotemark[1]
                \end{equation*}
                \footnotetext[1]{%
                    \begin{align*}
                            \int_{-\infty}^{\infty}e^{-ax^2} \difform x &=
                            \left(\int_{-\infty}^{\infty} \int_{-\infty}^{\infty}e^{-a\left(x^2 + y^2\right)}\difform x \difform y\right)^{1/2} \\
                            &= \left(\int_{0}^{2 \pi} \int_{0}^{\infty} r e^{-a r^2} \difform r \difform \theta\right)^{1/2} = \left(\frac{\pi}{a}\right)^{1/2}
                    \end{align*}
                }
            에서 $K_x = {\left(m / 2 \pi kT \right)}^2$이고
                \begin{equation*}
                    f\left( v_x \right) = {\left( \frac{m}{2 \pi k T} \right)}^{1/2} e^{-mv_x / 2kT}
                \end{equation*}
            을 만족한다. $f\left(v_y\right)$와 $f\left(v_z\right)$ 또한 마찬가지이다. 따라서 
                    \begin{align*}
                        &f\left(v_x\right) f\left(v_y\right) f\left(v_z\right) \difform {v_x} \difform {v_y} \difform {v_z} \\
                        &= 
                        {\left(\frac{m}{2 \pi kT}\right)}^{3/2} e^{-m\left({v_x}^2 + {v_y}^2 + {v_z}^2\right)/2kT} \difform v_x \difform v_y \difform v_z \\
                        &={\left(\frac{m}{2\pi k T}\right)}^{3/2} e^{-mv^2 /2kT} \difform v_x \difform v_y \difform v_z
                    \end{align*}
            을 만족하고, $\left(x,y,z\right)$를 $\left(r, \theta, \phi\right)$로 바꿀 때의 Jacobian을 고려하면
            \begin{equation*}
                    \difform x \difform y \difform z = r^2 \sin{\theta} \difform r \difform \theta \difform \phi
            \end{equation*}
            이므로, $\difform v_x \difform v_y \difform v_z = 4 \pi v^2 \difform v$를 만족한다. 
            \end{obs}
            따라서 분자의 속도에 대한 
            \textbf{Maxwell-Boltzmann 분포 (Maxwell-Boltzmann distribution)}는 다음과 같다:
                \begin{thm}[Maxwell-Boltzmann 분포]
                    \begin{align*}
                        f \left( v \right) &= 4 \pi {\left(\frac{m}{2\pi kT} \right)}^{3/2} v^2 e^{-mv^2 / 2kT} \\
                        &=4 \pi \left( \frac{M}{2\pi RT} \right)^{3/2} v^2 e^{-Mv^2 / 2RT}
                    \end{align*}
                \end{thm}
            이때 $R = N_A k$이고, $M$은 분자량이다.
            \par $\displaystyle\left\langle v^n \right\rangle = \int_{0}^{\infty} v^n f \left(v\right) \difform v$이고, 이를 통해 구한 기체 분자의 평균 속력은 다음과 같다:
            \begin{obs}[기체 분자의 평균 속력]
                \begin{align*}
                    v_{\mathrm{rms}} &= \left< v^2 \right>^{1/2} = \sqrt{\frac{3RT}{M}} \\
                    v_{\mathrm{mean}} &= \left< v \right> = \sqrt{\frac{8RT}{\pi M}} = \sqrt{\frac{8}{3 \pi}} \times v_{\mathrm{rms}}
                \end{align*}
            \end{obs}
            또한, 최빈 속도는 Maxwell-Boltzmann 분포의 미분계수가 $0$이 되는 지점이므로,
            \begin{obs}[기체 분자의 최빈 속력]
                \begin{equation*}
                    v_{\mathrm{mp}} = \sqrt{\frac{2RT}{M}} = \sqrt{\frac{2}{3}} \times v_{\mathrm{rms}}
                \end{equation*}
            \end{obs}
            \renewcommand{\thefootnote}{\arabic{footnote}}
            를 만족한다. 같은 분자끼리의 평균 상대 속도는 $v_{\mathrm{rel}} = \sqrt{2} \times v_{\mathrm{mean}}$으로 정의되고, 서로 다른 분자끼리의 
            평균 상대 속도는 \textbf{환산 질량(Reduced mass)} $\displaystyle\mu = \frac{m_A m_B}{m_A + m_B}$에 대해
                \begin{equation*}
                    v_{\mathrm{rel}} = \left(\frac{8kT}{\pi \mu}\right)^{1/2}
                \end{equation*}
            을 만족한다.
            \par 따라서 앞의 \ref{pvconst}에 $v_{\mathrm{rms}}$를 대입하면 이상 기체 상태방정식을 유도할 수 있다.
            \par 기체 분자가 진행할 때, 기체 분자의 반지름을 $d$라 할 때 기체 분자의 중심으로부터 반경 $2d$ 내에 있는 분자들은 
            기체 분자와 충돌한다. 이때 $\Delta t$ 동안 진행할 때 충돌하는 횟수는 개수 밀도 $\mathscr{N} = N / V$에 대해 
            $\mathscr{N}\sigma v_\mathrm{rel} \Delta t$이고, 충돌 빈도는 $z = \sigma v_\mathrm{rel} \mathscr{N}$이다. 이때 $\sigma$를 
            \textbf{충돌 단면적(Collision cross-section)}이라 한다. 여기에 이상 기체 상태방정식을 이용하여 $\mathscr{N} = \frac{N}{V} 
            = \frac{nN_A}{V} = \frac{pN_A}{RT} = \frac{p}{kT}$를 대입하면, 다음과 같은 식을 유도할 수 있다.
            \begin{obs}[충돌 빈도]
                \begin{equation*}
                    z = \frac{\sigma v_{\mathrm{rel}} p}{kT}
                \end{equation*}
            \end{obs}
            따라서 \textbf{평균 자유 경로(Mean free path)} $\lambda$는 다음과 같이 정의된다:
            \begin{defn}[평균 자유 경로]
                \begin{equation*}
                    \lambda = \frac{v_\mathrm{rel}}{z} = \frac{kT}{\sigma p}
                \end{equation*}
            \end{defn}
        \section{실제 기체}
            \hspace{\parindent} 실제 기체에서는 분자 간 상호작용을 고려하게 된다. 이때 거리가 너무 가까우면 반발력이, 거리가 멀면 인력이 작용하고, 따라서 
            특정 온도(\textbf{임계 온도(Critical temperature), }$T_c$) 이하에서는 액화된다. 등온 곡선에서 이는 그래프의 불연속으로 
            나타난다. \textbf{압축도(Compression factor)} $Z$는 실제 기체의 몰부피 $V_m$과 이상 기체의 몰부피 $V_m^\circ$에 대해 
            다음과 같이 정의된다:
            \begin{defn}[압축도]
                \begin{equation*}
                    Z = \frac{V_m}{V_m^{\circ}}
                \end{equation*}
            \end{defn}
            따라서 실제 기체의 상태방정식은 다음과 같다:
            \begin{obs}[실제 기체의 상태방정식]
                \begin{equation*}
                    pV_m = RTZ
                \end{equation*}
            \end{obs}
            이상 기체에서는 $Z = 1$을 만족하고, $Z < 1$일 때에는 인력이 우세하다. 반면, $Z > 1$일 때에는 척력이 우세하다.
            \par 실제 기체에서 다음과 같이 쓸 수 있다:
            \begin{obs}
                \begin{equation*}
                    pV_m = RT \left(1 + B^{\prime} p + C^{\prime} p^2 + \cdots \right)
                \end{equation*}
            \end{obs}
            이를 부피에 대해 전개하면 다음과 같은 \textbf{비리얼 상태방정식(Virial equation of state)}을 유도할 수 있다:
            \begin{obs}[비리얼 상태방정식]
                \begin{equation*}
                    pV_m = RT \left( 1 + \frac{B}{V_m} + \frac{C}{{V_m}^2} + \cdots \right)
                \end{equation*}
            \end{obs}
            대부분의 경우에서, $\left\vert C / {V_m}^2 \right\vert << \left\vert B / V_m \right\vert$이다. 또한, 다음과 같은 관계식도 
            유도된다:
            \begin{obs}
                    $V_m \rightarrow \infty$ 이면 $\displaystyle\frac{dZ}{d\left(1/V_m\right)} \rightarrow B$ 
            \end{obs}
            $p \to 0$일 때, $Z \to 1$이며 $\displaystyle\frac{dZ}{dp} \to 0$을 만족하는 온도를 \textbf{Boyle 온도($T_B$)}라 한다.
            \par 압력-부피 곡선에서 등온 곡선이 불연속이 되기 시작하는 지점을 \textbf{임계 온도(Critical temperature, $T_c$)}라 한다. 
            이 지점을 \textbf{임계점(Critical point)}이라 하고, 이때의 압력을 \textbf{임계 압력(Critical pressure, $p_c$)}, 
            이때의 부피를 \textbf{임계 부피(Critical volume, $V_c$)}라 한다.$p_c$, $V_c$, $T_c$를 모두 일컬어 물질의 
            \textbf{임계 상수(Critical constant)}라 한다. $T \geq T_c$일 때 물질은 액체 상태로 존재하지는 않으나, 압력이 높을 때 이를 
            \textbf{초임계 유체(Supercritical fluid)}라 한다.
            \par Johannes Diderik van der Waals는 다음과 같은 과정을 통해 \textbf{van der Waals 상태방정식(equation)}을 유도하였다:
            \begin{obs}
            기체 분자 자체의 반발력으로 인해, 기체 분자가 이동할 수 있는 부피는 $V-nb$로 제약된다. 이때 $b \approx 4 V_{\textrm{molecule}} N_A$
            이다. 한편, 기체 분자끼리의 인력으로 인해, 기체 분자가 가하는 압력은 $a \frac{n^2}{V^2}$만큼 감소한다.
            \end{obs}
            따라서 최종 방정식은 다음과 같다:
            \begin{law}[van der Waals 상태방정식]
                \begin{equation*}
                    p = \frac{nRT}{V-nb} - a \frac{n^2}{V^2}
                \end{equation*}
            \end{law}
            상수 $a$와 $b$는 \textbf{van der Waals 계수(coefficient)}라 한다. $a$는 분자 간 인력과 관련된 계수이고, $b$는 분자 간 반발력과 
            관련된 계수이다. 이를 몰부피로 표현하면 다음과 같다.
                \begin{equation*}
                    p = \frac{RT}{V_m - b} - \frac{a}{{V_m}^2}
                \end{equation*}
            van der Waals 방정식을 다르게 표현하면
                \begin{equation*}
                    {V_m}^3 - \left(b+\frac{RT}{p}\right) {V_m}^2 + \left(\frac{a}{p}\right) V_m + \frac{ab}{p} = 0
                \end{equation*}
            으로 표현할 수 있으나, $T > T_c$에서 "압력이 증가할 때 부피가 증가하는" 그래프가 그려진다. 이 부분을 \textbf{van der Waals loop}이라 
            하고, 이를 보완하기 위해 압력-부피 그래프에서 압력축에 수직인 직선을 그어 van der Waals loop과 이 직선 사이의 적분이 $0$이 되게 
            하는 \textbf{Maxwell construction}으로 처리한다. \\
            van der Waals 상태방정식에서, $T_c$일 때 다음을 만족한다:
                \begin{align*}
                    \frac{dp}{dV_m} &= \frac{RT}{\left( V_m - b \right)^2} + \frac{2a}{{V_m}^3} = 0 \\
                    \frac{d^2 p}{d{V_m}^2} &= \frac{2RT}{\left( V_m - b \right)^3} - \frac{6a}{{V_m}^4} = 0
                \end{align*}
            따라서, 다음을 만족한다:
            \begin{law}[임계 압력, 임계 부피, 임계 온도]
                \begin{equation*}
                    p_c = \frac{a}{27b^2},\ V_c = 3b, \ T_c = \frac{8a}{27bR}
                \end{equation*}
            \end{law}
            이때 \textbf{임계 압축도(Critical compression factor, $Z_c$)}는 $Z_c = \frac{p_c V_c}{RT_c} = \frac{3}{8}$을 
            만족한다.
            \par 다음과 같이 \textbf{환산 변수(Reduced variables)}를 정의하자:
            \begin{defn}[환산 변수]
                \begin{equation*}
                    V_r = \frac{V_m}{V_c}, \ p_r = \frac{p}{p_c}, \ T_r = \frac{T}{T_c}
                \end{equation*}
            \end{defn}
            이때, 실제 기체는 다음과 같은 관계식을 잘 만족한다:
            \begin{law}[대응 상태 원리]
                \begin{equation*}
                    p_r = \frac{8T_r}{3V_r - 1} - \frac{3}{{V_r}^2}
                \end{equation*}
            \end{law}
            이를 \textbf{대응 상태 원리(Principle of corresponding states)}라 한다.